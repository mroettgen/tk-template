\documentclass[12pt,twoside]{article}
\usepackage[utf8]{inputenc}
\usepackage[T1]{fontenc}
\usepackage[german]{babel}
\usepackage{csquotes}

% Konfiguration der Anführungszeichen
\usepackage[left = \glqq{},right = \grqq{},leftsub = \glq{},rightsub = \grq{}]{dirtytalk}

\usepackage{graphicx}
\usepackage[graphicx]{realboxes}

% Fügt die Möglichkeit hinzu, in Text-Tabellen mit L, C, und R das Text-Alignment (Links, Mitte, Rechts) zu bestimmen
\usepackage{array}
\newcolumntype{L}[1]{>{\raggedright\let\newline\\\arraybackslash\hspace{0pt}}m{#1}}
\newcolumntype{C}[1]{>{\centering\let\newline\\\arraybackslash\hspace{0pt}}m{#1}}
\newcolumntype{R}[1]{>{\raggedleft\let\newline\\\arraybackslash\hspace{0pt}}m{#1}}

% Andere spannende Tabellen-Pakete
\usepackage{xcolor}
\usepackage{colortbl}
\usepackage{booktabs}

\usepackage{pdfpages}

% Legt die Schriftart auf Helvetica fest.
% WICHTIG: Das ist optisch nah an der Template-Vorgabe Arial, aber nicht komplett identisch.
\usepackage[scaled]{helvet}
\renewcommand{\familydefault}{\sfdefault}

% Kein Plan mehr, was das tut. Gerne löschen und schauen, ob was kaputt geht
\renewcommand\labelenumi{(\theenumi)}

% Konfiguration vom Zitationsstil
\usepackage[style=authoryear-icomp,maxcitenames=2,backend=biber]{biblatex}
\DefineBibliographyStrings{german}{ 
   andothers = {{et\,al\adddot}},             
}
\addbibresource{bibliography.bib}

% Ganz viel Abstandsgedöns
\usepackage{etoolbox}
\AtBeginEnvironment{quote}{\small\setstretch{.25}}

\usepackage{titlesec}
\titlespacing\section{0pt}{12pt plus 4pt minus 2pt}{0pt plus 2pt minus 2pt}
\titlespacing\subsection{0pt}{12pt plus 4pt minus 2pt}{0pt plus 2pt minus 2pt}
\titlespacing\subsubsection{0pt}{12pt plus 4pt minus 2pt}{0pt plus 2pt minus 2pt}
\titleformat*{\section}{\LARGE\bfseries}
\titleformat*{\subsection}{\Large\bfseries}
\titleformat*{\subsubsection}{\large\bfseries}

\usepackage{caption} 
\captionsetup[table]{skip=10pt}

\usepackage{enumitem}
\setlist[itemize]{noitemsep, topsep=0pt}
\setlist[enumerate]{noitemsep, topsep=0pt}

% Baut aus Gründen ein helleres Grau aus einem dunklen Grau?
\usepackage{color}
\definecolor{hgray}{gray}{0.5}

% Konfiguration vom hyperref-Package, duh.
\usepackage[hidelinks,
    pdfpagelabels,
    pdfstartview = FitH,
    bookmarksopen = true,
    bookmarksnumbered = true,
    linkcolor = black,
    plainpages = false,
    hypertexnames = false,
    citecolor = black
]{hyperref}

\usepackage{setspace}

% Stellt die Seitenränder richtig ein
\usepackage{geometry}
\geometry{
  left=2.5cm,
  right=2.5cm,
  top=2.5cm,
  bottom=2cm,
  bindingoffset=0mm
}

\usepackage{parskip}

\makeatletter
\title{Abschlussarbeit 4}\let\Title\@title     %   Titel der Arbeit eintragen
\makeatother

% Seitenstil mit Titel der Arbeit oben und dem grauen Strich.
% Außerdem Seitenzahlen abwechselnd rechts und links.
% WICHTIG: Die abwechselnden Seitenzahlen weichen vom Template ab. Find ich aber sinnvoller und schöner so.
\usepackage{fancyhdr}
\usepackage{afterpage}
\fancypagestyle{MRstyle}{
    \fancyhf{}
    \fancyhead[L]{\textit{\textcolor{hgray}{\Title}}}
    \fancyfoot[LE,RO]{\textcolor{hgray}{\thepage}}
}

% Ermöglicht es, mit dem Befehl \fakesection{Name} Kapitelüberschriften im Anhang oben rechts zu platzieren.
% Hab ich für meine MA nicht benutzt, kann sein, dass das kaputt ist. Ist aber auch nicht Bestandteil des offiziellen Templates.
\fancypagestyle{appendix}{
    \fancyhf{}
    \fancyhead[L]{\textit{\textcolor{hgray}{\Title}}}
    \fancyhead[R]{\textcolor{hgray}{\rightmark}}
    \fancyfoot[LE,RO]{\textcolor{hgray}{\thepage}}
}

\newcommand\fakesection[1]{% 
    \markboth{#1}{#1}
}
    
\usepackage[hang]{footmisc}

% Fix, falls zu viele Bilder/Tabellen in der Arbeit vorkommen
\extrafloats{300}
\maxdeadcycles=200

% Package wird verwendet, um nicht alles in eine riesige Latex-Datei schreiben zu müssen: https://www.overleaf.com/learn/latex/Multi-file_LaTeX_projects#The_subfiles_package
\usepackage{subfiles}

% Kleinerer Zeilenabstand, damit die ersten Seiten nicht so hässlich aussehen. 
% WICHTIG: Kein Plan, wie das im Word Template ist.
\setstretch{1.2}

% Das Dokument geht hier los:

\begin{document}
\pagestyle{MRstyle}
 
\begin{titlepage}
    
    % Bei Bedarf natürlich nicht vergessen das Institut anzupassen 🤠
    \large
    RWTH Aachen\\
    Institut für Sprach- und Kommunikationswissenschaft\\
    Professur für Textlinguistik und Technikkommunikation\\
    Prof. Dr. E.-M. Jakobs
    
    \vspace{5cm}
    \Large
    \doublespacing{
        \textit{Masterarbeit\\} % <-- oder eben Bachelorarbeit. Oder Hausarbeit. Oder Praktikumsbericht.
        \textbf{\Title}
    }
    
    \vspace{7cm}
    \normalsize
    \setstretch{1.2}
    vorgelegt von:\\
    Max Mustermann (Matrikel-Nr.: 123456)\\ % Hier Name, Matr. Nr. etc. einfügen
    Tel.-Nr.: +49 1234 567891011\\                   %
    max.mustermann@rwth-aachen.de           %
    
    \vfill
    
    Aachen, \today    
\afterpage{\cfoot{\textcolor{hgray}{\thepage}}}
        
\end{titlepage}

% Seite einfügen für die doppelseitige Bedruckung (dieser Block hier kommt noch paar Mal vor)
\pagenumbering{gobble}
\newpage
\thispagestyle{empty}
\mbox{}
\newpage

\pagenumbering{Roman}

\tableofcontents
\clearpage

\listoffigures
\listoftables
\clearpage

% Wie versprochen noch eine blanke Seite. Könnte passieren, dass man das für den Druck anpassen muss.
% Aber das sollte man nach dem Studium ja hinkriegen.
\pagenumbering{gobble}
\newpage
\thispagestyle{empty}
\mbox{}
\newpage

% Zeilenabstand auf ungefähr das Word-Template.
% WICHTIG: Wenn das jemand super pingelig nachmisst, kann es schon sein, dass sich der Zeilenabstand von Word unterscheidet.
% Das ist aber eigentlich ein Word-Problem, siehe https://tex.stackexchange.com/questions/65849/confusion-onehalfspacing-vs-spacing-vs-word-vs-the-world
\linespread{1.3}
\section*{Zusammenfassung}
{\small
    Hier kommt das Abstract hin
}

% Ha! Schon wieder eine blanke Seite.
\pagenumbering{gobble}
\newpage
\thispagestyle{empty}
\mbox{}
\newpage


\clearpage
\pagenumbering{arabic}

\subfile{sections/01_einleitung}

\clearpage
\subfile{sections/02_literatur}

\clearpage
\subfile{sections/03_methodik}

\clearpage
\subfile{sections/04_ergebnisse}

\clearpage
\subfile{sections/05_diskussion}

\clearpage
\subfile{sections/06_methodenreflexion}

\clearpage
\subfile{sections/07_fazit}

\cleardoublepage
\pagenumbering{Roman}
\appendix

% Schafft ein bisschen mehr Raum für den ganzen nutzlosen Kram im Anhang.
% WICHTIG: Ist im offiziellen Template nicht vorgesehen.
\newgeometry{
    left=2.5cm,
    right=2.5cm,
    top=2cm,
    bottom=2cm,
    bindingoffset=0mm
    }
\pagestyle{appendix}

\clearpage
\section{Literaturverzeichnis}
\pagestyle{MRstyle}
\setcounter{biburllcpenalty}{7000}
\setcounter{biburlucpenalty}{8000}
\printbibliography[heading=none]
\clearpage


\section{Tabellen}

%\input{Tabellen/tolletabelle}

\clearpage
% ⤵ so können Tabellen über externe Latex-Dateien eingefügt werden: https://www.overleaf.com/learn/latex/Management_in_a_large_project#Inputting_and_including_files
%\input{Tabellen/01_Screening_Daten}
\clearpage


\pagestyle{plain}
\pagenumbering{gobble}

% ⤵ Eidesstattliche Versicherung als PDF einbinden.
%\includepdf[pagecommand={}]{Eidesstattliche Versicherung.pdf}
\newpage

\end{document}